\documentclass[10pt,twocolumn,letterpaper]{article}

\usepackage{cvpr}
\usepackage{times}
\usepackage{epsfig}
\usepackage{graphicx}
\usepackage{amsmath}
\usepackage{amssymb}
\usepackage{subcaption}

% Include other packages here, before hyperref.

% If you comment hyperref and then uncomment it, you should delete
% egpaper.aux before re-running latex.  (Or just hit 'q' on the first latex
% run, let it finish, and you should be clear).
\usepackage[breaklinks=true,bookmarks=false]{hyperref}

\cvprfinalcopy

\def\cvprPaperID{****} % *** Enter the CVPR Paper ID here
\def\httilde{\mbox{\tt\raisebox{-.5ex}{\symbol{126}}}}

% Pages are numbered in submission mode, and unnumbered in camera-ready
%\ifcvprfinal\pagestyle{empty}\fi
\setcounter{page}{1}
\begin{document}

%%%%%%%%% TITLE
\title{
    Super Convergence: Very Fast Training of Residual Networks Using Large
    Learning Rates\\
    \large ICLR Reproducibility Challenge}

\author{Keivaun Waugh\\
University of Texas at Austin\\
{\tt\small keivaunwaugh@gmail.com}
% For a paper whose authors are all at the same institution,
% omit the following lines up until the closing ``}''.
% Additional authors and addresses can be added with ``\and'',
% just like the second author.
% To save space, use either the email address or home page, not both
\and
Paul Choi\\
University of Texas at Austin\\
{\tt\small choipaul96@gmail.com}
}

\maketitle
%\thispagestyle{empty}

%%%%%%%%% ABSTRACT
\begin{abstract}
In this paper, we aim to evaluate the reproducibility of the experiments
    detailed in the paper ``Super Convergence: Very Fast Training of Residual
    Networks Using Large Learning Rates'' \cite{SuperConvergence}
\end{abstract}

%%%%%%%%% BODY TEXT
\section{Introduction}
Paul

\subsection{Target Questions}
%------------------------------------------------------------------------------

\section{Method}
\label{sec:method}
Due to time and hardware constraints (lack of multiple GPUs), we were unable to replicate all of the experiments that were listed in the paper. Therefore, we had to choose which experiments from which we could glean the most. 

\subsection{Implementation Details}
Discuss things like batch size that we were limited by because we only had
one GPU to run things on at a time (8GB of memory)

%------------------------------------------------------------------------------

\section{Experiments}
Keivaun
\subsection{Methodology}
%------------------------------------------------------------------------------

\section{Conclusion}
\label{sec:conclusion}
Paul

\subsection{Cost of Reproduction}
What cost in terms of resources (computation, time, people, developemnt effort, communication with the authors).

{\small
\bibliographystyle{ieee}
\bibliography{bib}
}

\end{document}
